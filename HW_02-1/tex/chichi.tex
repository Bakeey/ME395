% !TEX root = ./main.tex
We get the displacement signal for the chichi earthquake:

\begin{figure}[ht!]
    \centering
    \includegraphics*[width=0.75\columnwidth]{chichi_displacement.pdf}
    \caption{Chichi Earthquake Displacement Profile}
\end{figure}

Next, we get the velocity profile using a differentiation filter. 
We use the forward difference (Euler forward) and central difference filter and can compare these two
methods to the given velocity measurement signal in figure \ref{chichi_velocity}. 

\begin{figure}
    \centering
    \includegraphics*[width=0.75\columnwidth]{chichi_velocity.pdf}
    \caption{Chichi earthquake velocity, measurement and computed via filter.}
    \label{chichi_velocity}
\end{figure}

We see that the computed velocity signal seems to be generally in agreement with the measured velocity
signal, and plot the difference of the computed signal (differentiated from displacement using filter) 
to the measured signal in figure \ref{chichi_velocity_error}. 
Unsurprisingly, central difference is more accurate than forward difference.

\begin{figure}
    \centering
    \includegraphics*[width=0.75\columnwidth]{chichi_velocity_error.pdf}
    \caption{Chichi earthquake velocity error, comparing computed velocity differentiated from 
             displacement to measurmeent velocity.}
    \label{chichi_velocity_error}
\end{figure}

\clearpage
Lastly, we can derive the acceleration signal by differentiating again. 

First, we perform differentiation using the central difference method on the velocity data 
computed from the displacement profile using central difference.

Secondly, we perform backward difference on the forward-difference velocity profile.

Third, we obtain the acceleration profile directly from the displacement measurement using second-order 
finite difference:

\begin{equation}
    \frac{d^2 u}{d t^2}\left(t_j\right)=\frac{u\left(t_{j+1}\right)-2 u\left(t_j\right)+u\left(t_{j-1}\right)}{\left(t_j-t_{j-1}\right)^2}
    \label{eq:1}
\end{equation}

We plot all of thse derived acceleration signals in figure \ref{chichi_acceleration}. 
Similarly to figure \ref{chichi_velocity_error}, compare the computed acceleration signal to the given 
acceleration measurement in figure \ref{chichi_acceleration_error}.

We notice that the acceleration error for the signal using backwards differentiation is the same as for the 2nd 
order central difference method. This is because as seen in the handwritten homework, performing differentiation twice 
using forward and backward difference yields the same result as the 2nd order central difference method seen in equation \ref{eq:1}.

Lastly, we see that performing first order central difference twice to receive a second order derivative 
results in a larger error. This is because since approximating the 1st order derivative via central differences can be written as
$$
u^{\prime}(x) \approx \frac{u(t_{j+1})-u(t_{j-1})}{(t_{j+1}-t_{j-1})}
$$
What is the main issue with applying again a central difference to compute $u^{\prime \prime}(t_j)$ is:
$$
\frac{d u^\prime}{dt}(x)=\frac{u^{\prime}(t_{j+1})-u^{\prime}(t_{j-1})}{(t_{j+1}-t_{j-1})} \approx \frac{u\left(t_{j+2}\right)-2 u\left(t_j\right)+u\left(t_{j-2}\right)}{(t_{j+1}-t_{j-1})^2}
$$
Which is less accurate than applying forward and backward difference!


\newpage

\begin{figure}
    \centering
    \includegraphics*[width=0.75\columnwidth]{chichi_acceleration.pdf}
    \caption{Chichi earthquake acceleration, measurement and computed via filter.}
    \label{chichi_acceleration}
\end{figure}

\begin{figure}
    \centering
    \includegraphics*[width=0.75\columnwidth]{chichi_acceleration_error.pdf}
    \caption{Chichi earthquake acceleration error, comparing computed acceleration to measurmeent acceleration.}
    \label{chichi_acceleration_error}
\end{figure}
\clearpage